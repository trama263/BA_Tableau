\chapter{Grundlagen der Aussagenlogik}\label{sec:Grundlagen}

%In diesem Kapitel werden wir uns mit der Grundlagen der Aussagenlogik beschäftigen.
In diesem Kapitel werden die Grundlagen der Aussagenlogik behandelt. Die Aussagenlogik ist ein Zweig der formalen Logik, der die Beziehungen zwischen Aussagen und Aussagenverbindungen untersucht. Aussagen sind abstrakte Begriffe, auch Propositionen genannt, die in der Alltagssprache durch Sätze ausgedrückt werden. Dabei kommt es in der Aussagenlogik nicht auf den konkreten Inhalt der Aussagen an, sondern nur auf die Entscheidung, ob eine Aussage wahr oder falsch ist.

\begin{ex} \label{Beispiel 4.0} \end{ex}
\begin{enumerate}
\item Der Mars ist ein Planet. 
\item  Der Mond ist ein Planet. 
\end{enumerate}

drücken zwei verschiedene Aussagen aus, wovon die erste wahr und die zweite falsch ist.





