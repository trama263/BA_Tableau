\section{Erfüllbarkeit und Tautologien}
Grundsätzlich kann man Formeln danach klassifizieren, ob überhaupt passende Belegungen
existieren, die sie wahr oder falsch machen.
\begin{defi} \label{Definition 2.38} \end{defi} 
Eine Formel A heißt: 
\begin{itemize}																					
\item \textit{erfüllbar}, falls sie unter wenigstens einer Belegung wahr ist.
\item \textit{Tautologie}, falls sie für jede passende Belegung wahr ist.
\item \textit{unerfüllbar}, falls es keine passende Belegung wahr gibt.
\end{itemize}
\begin{sa} \label{Satz 2.39} \end{sa}
Eine Formel $A$ ist gültig genau dann wenn $\neg A$ unerfüllbar ist.