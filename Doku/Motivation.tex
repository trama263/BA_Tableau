\chapter{Einleitung}
Aussagenlogik ist immer ein grundlegendes Thema in der Informatik. Dabei ist die Ermittlung der Allgemeingültigkeit oder Erfüllbarkeit einer aussagenlogischen Formel unverzichtbar. Es gibt viele Möglichkeiten, dies zu tun, wobei die Verwendung der Wahrheitstafel recht aufwändig sein kann. Die Äquivalenzumformung liefert auch kein Verfahren, das sagt, welche Regel anzuwenden ist \cite{Schaefer}.
%eine häufig verwendete Methode die Verwendung der Äquivalenzen ist. Diese Methode bietet jedoch keine Möglichkeit zu wissen, welche Regeln gelten, so dass der Verwender ein umfangreiches Training benötigt, um dies zu lösen.
Auch diese Methode ist schwer zu programmieren. Eine andere Methode ist einfacher und intuitiver:  Eine Formel der Aussagenlogik kann als ein semantisches Tableau nach syntaktischen Regeln konstruiert werden. Aus dem derart konstruierten Tableau kann dann ermittelt werden, ob eine Formel erfüllbar oder allgemeingültig ist. Diese Methode kann auch einfach programmiert werden. Dies sind jedoch neue Inhalte, die seit dem WS 2017-2018 in den Lehrplan der Informatik 1 aufgenommen wurden. Um den Studierenden den Erwerb dieser Methode zu erleichtern, wird daher die Idee einer Webanwendung gebildet.

Das Hauptziel dieser Arbeit ist es, eine Webanwendung zu entwickeln. Es ermöglicht dem Benutzer, eine aussagenlogische Formel einzugeben und die Allgemeingültigkeit oder Erfüllbarkeit der Formel zu überprüfen. Zuerst prüft die Anwendung, ob die Formel syntaktisch korrekt ist. Wenn dies nicht der Fall ist, wird die Anwendung den Benutzer informieren, um es zu beheben. Die Anwendung zeigt dann die Konstruktionen für die Eingabeformel und für das Tableau an. Schließlich wird das Prüfergebnis auf dem Bildschirm angezeigt.

Der Inhalt dieser Arbeit gliedert sich in 6 Hauptteile. Die erste besteht darin, das Problem und die Anforderungen der Anwendung zu analysieren (Kapitel \ref{sec:Analyse}). Als nächstes lernt man etwas über die Grundlagen der Aussagenlogik (Kapitel \ref{sec:Grundlagen}) und man wählt die geeigneten Frameworks (Kapitel \ref{sec:Frameworks}). Als nächstes soll die Architektur der Anwendung (Kapitel \ref{sec:Architektur}) erstellt und umgesetzt werden (Kapitel \ref{sec:Implementierung}). Abschließend: Anwendungstests und Test-Bewertung (Kapitel \ref{sec:Softwaretest})

\cleardoublepage
