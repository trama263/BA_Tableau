\section{FormulaGrammar.g4}
Aus dem Unterabschnitt \ref{subsec:FormelGrammatik} kann eine Formel als Strings, die über eine kontextfreie formale Grammatik generiert wird, definiert werden. Das Alphabet der Formel, das in der Definition \ref{Definition 2.1} definiert wird, wird wie folgt implementiert.
\begin{itemize}
\item Die Aussagenvariable kann beliebige große und kleine Buchstaben enthalten.

\begin{lstlisting}[language=Grammar, basicstyle=\scriptsize]
ATOM: [a-zA-Z];
\end{lstlisting}
\item Basierend auf dem Unterabschnitt \ref{subsec:Notation} werden die folgenden Notationen der Junktoren verwendet:

\begin{lstlisting}[language=Grammar, basicstyle=\scriptsize]
NEG : '!';    // negation
IMP : '=>';   // implication		
EQU: '<=>';   // equivalence
XOR: '^';     // exclusive or
AND : '&';    // conjunction
NAND : '!&';  // nand
OR : '|';     // disjunction
NOR : '!|';   // nor
\end{lstlisting}

\item Die Konstanten kann 0, 1 , true oder false (in ``case-insensitive'') sein.

\begin{lstlisting}[language=Grammar, basicstyle=\scriptsize]
TRUE: (T R U E| '1'); 
FALSE: (F A L S E| '0');
fragment T : [tT]; // match either an 't' or 'T'
fragment R : [rR];
fragment U : [uU];
fragment E : [eE];
...
\end{lstlisting}
\end{itemize}

ANTLR löst Mehrdeutigkeiten zugunsten der zuerst genannten Alternative. Aus den Bindungsregeln in dem Unterabschnitt \ref{subsec:Bindungskonventionen} hat 
die Regel \textit{expr} z.B die Konjuktionsalternative vor der Disjunktionsalternative. Die Produktionen von der Grammatik der aussagenlogischen Formeln wird wie folgt implementiert: 

\begin{lstlisting}[language=Grammar, basicstyle=\scriptsize]
stat :	expr EOF;
expr :	NEG expr                     # Negation
		| expr op=('&'|'!&') expr        # AndNand
		| expr op=('|'|'!|') expr        # OrNor		
		| expr '=>' expr                 # Implication	
		| expr op=('<=>'|'^') expr       # EquXor							
		| ATOM                           # Atom
		| TRUE                           # True
		| FALSE                          # False
		| '(' expr ')'                   # Parens
		;
\end{lstlisting}


Dazu sind ``\# Negation'', ``\# AndNand'' usw. die Bezeichnungen von der Alternativen der Regel \textit{expr}. Diese Bezeichnungen erscheinen am rechten Rand von Alternativen und beginnen mit dem \# -Zeichen.